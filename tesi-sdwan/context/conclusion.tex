\section{Conclusion and future work}

The tests results show how a reactive strategy can effectively route traffic coming from a variety of applications. \\
Even when compared to an ideal strategy, the increase in costs in using are strategy sits at around 10\%.

While the idea can seem promising, further work can be done in the analysis of the proposed strategy.

\subsection{Performance analysis}
Throughout our testing we considered a strategy that is executed without any time constraint: this is unrealistic.
In practice a strategy that is too time expensive to compute could lead to increased delay just to determine where to send the incoming packets. \\
In our tests we executed our strategy at a fixed time rate to compute the split ratio of the generated packets among all available links, a possible area of research could be keeping a given split ratio for a set amount of time and seeing how the quality of the overall transmission change.


\subsection{Improving the delay function}
As we discussed in section \ref{estimating_the_delay} the delay of sending a packet on a given link is computed with a fixed form function that is based on the previous usage of a given interface. \\
Some other approaches can be tried to further improve the strategy, some possible advancement could include:
\begin{itemize}
	\item ML-Driven models: instead of basing the estimation of the delay on only the current run, with a wider collection of data it could be possible to implement an estimation based on a machine learning model \cite{ml_traffic_predictions}.
\end{itemize}


\subsection{Extending the available traffic models}
The simulator easily allows to define arbitrary applications. \\
This opens a wide variety of possiblity in testing different traffic models and how the efficency of different strategies changes. \\
Real traffic data could be recorded, encoded and replayed to test how the strategy would behave in a real life scenario.