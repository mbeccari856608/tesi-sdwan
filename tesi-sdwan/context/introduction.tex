\section{Introduction}

\subsection{WANs and their limitation}

Large companies and organizations often rely on  WANs (\textit{W}ide \textit{A}rea \textit{N}etworks) to connect their branches. This is due to the higher level of guarantees that than a WAN can offer over the traditional connectivity used by the average user. The main drawback in traditional WANs lies in the sharp increase in cost, while different solutions are available (as we will detail in section \ref{wans}) they usually are orders of magnitude more expensive than a traditional ISP (\textit{Internet Service Provider}) connectivity.


\subsection{The hybrid approach of SD-WANs}

One recent solution in decreasing the cost of creating WANs is mixing traditional WAN approaches with the generalized connectivity offered to the average customer (i.e PON, 4G).
Different network applications (e.g. VoIP, emails, web navigation \dots) have different requirements in terms of the quality of network they rely on; a VoIP application will require a lower delay than requesting a static web page.
The basic idea is that traditional WAN solutions are used only when strictly necessary; other types of cheaper connections can be used instead in all other cases.

\subsection{The basic structure of a SD-WAN}
At a very high level a SD-WAN works by installing a CPE (\textit{Customer} \textit{Premises} \textit{Equipment}) in each branch of the business.

The CPEs is connected to the branch's LAN and to multiple generalized connectivity networks (i.e multiple contracts are made with different ISPs).

The traffic generated and received by the LAN may then travel in different networks, potentially even from different ISPs.

An specialized piece of equipment, in this specific context called an \textit{SD-WAN Controller} is responsible to communicate to each CPE how the outgoing traffic should be managed (e.g two branches may be connected using a fiber cable, if the cable connection fails the controller will instruct the branches to communicate using a 4G network ).

\subsection{The challenges of SD-WAN}
The design of a SD-WAN network introduces many challenges:
\begin{enumerate}
	\item The placement and the amount of the controllers
	\item The choice of the links
	\item The policy used to determine on which links the outgoing traffic from each application should be directed.
\end{enumerate}
In this thesis we will propose a possibile solution for the third challenge.

\subsection{The scope of this thesis}
In this thesis we will explore a strategy to route the outgoing traffic in a way that:
\begin{enumerate}
	\item Respects the requirements of the application
	\item Minimizes the overall cost of the transmission
\end{enumerate}

To solve the cost minimization problem, we will start from a simplified scenario in which all the traffic generated by a given set of application is known beforehand.
Then we will explore a more realistic strategy that is executed periodically to account for the varying bandwidth requirements of each application.
