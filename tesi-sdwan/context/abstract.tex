\pagebreak
\thispagestyle{empty}
\hspace{0pt}
\vfill
\textbf{Abstract} \\
\vspace{0.2cm}
\begin{flushleft}
	Wide Area Networks (\textit{WANs})  are adopted by large companies and organizations to overcome the potential reliability issues that may arise by just using general connectivity (i.e the standard connections offered by Internet Service Providers (\textit{ISPs}), usually sold for personal use). \\
	However, while offering greater guarantees, setting up a full WAN connectivity is massively more expensive than using ISPs; some WANs may even require renting actual cables from ISPs.
	SD-WANs (\textit{Software Defined WAN}) offer an hybrid approach to find a middle ground between the reliability of standard WANs and the economic advantage of general connectivity. \\
	When using both standard WAN connectivity solutions and general connectivity traffic may be routed through the different interfaces according to the requirements of the source of the traffic itself. (i.e the more expensive interfaces are only used when strictly necessary) \\
	This paper tries to explore a strategy to route traffic across different kinds of interfaces to minimize the overall transmission cost, while still respecting the requirements of each source.
\end{flushleft}
\vfill
\hspace{0pt}
\pagebreak